\documentclass[a4paper,12pt]{article}
\usepackage[utf8]{inputenc}
\usepackage[T1]{fontenc}
\usepackage[polish]{babel}
\usepackage{amsmath, amssymb}
\usepackage{hyperref}
\usepackage{geometry}
\geometry{a4paper, margin=1in}

\title{Sprawozdanie – System Wirtualnej Kamery}
\author{Adrian Rybaczuk 318483}
\date{\today}

\begin{document}
\maketitle
\tableofcontents
\newpage

\section{Wstęp}

Projekt polega na implementacji systemu wirtualnej kamery. Celem jest przedstawienie sceny przy pomocy wirtualnej kamery. Użyłem do tego podejścia z wykorzystaniem 2 układów, układu kamery oraz układu sceny.

\section{Transformacja obiektu 3D}

W tej sekcji przedstawiamy szczegółowy opis procesu przekształcania obiektu 3D z jego pierwotnego umiejscowienia w scenie do ostatecznego obrazu na ekranie. Wszystkie operacje wykonujemy w przestrzeni jednorodnej (homogenicznej), co umożliwia łączenie transformacji translacji, rotacji oraz skalowania w jedną macierz.

\subsection{Krok 1: Umiejscowienie Obiektu na Scenie (Modeling)}
Obiekt w scenie jest pierwotnie opisany w swoim lokalnym układzie współrzędnych. Aby umiejscowić go w przestrzeni świata, stosujemy modelową macierz transformacji \( M \),
\[
M = T \cdot R \cdot S,
\]
gdzie:
\begin{itemize}
    \item \( T \) – macierz translacji,
    \item \( R \) – macierz rotacji,
    \item \( S \) – macierz skalowania.
\end{itemize}
Wszystkie współrzędne obiektu zapisujemy jako wektory jednorodne:
\[
P_{model} = \begin{bmatrix} x \\ y \\ z \\ 1 \end{bmatrix}.
\]
W wyniku przekształcenia mamy:
\[
P_{world} = M \cdot P_{model}.
\]

\subsection{Krok 2: Przelozenie Obiektu z Widoku Sceny na Widok Kamery (View Transformation)}
Następnie, aby przejść z układu świata (sceny) do układu kamery, stosujemy macierz widoku \( V \). Jest ona konstruowana na bazie pozycji kamery \( E \) (ang. eye), punktu, na który kamera patrzy, oraz wektorów definiujących orientację kamery.
Metoda \texttt{lookAt} definiuje trzy ortonormalne wektory:
\begin{itemize}
    \item \( \mathbf{u} \) – wektor określający prawą stronę kamery,
    \item \( \mathbf{v} \) – wektor skierowany do góry,
    \item \( \mathbf{n} \) – wektor przeciwległy do kierunku patrzenia.
\end{itemize}
Macierz widoku ma postać:
\[
V = \begin{bmatrix}
u_x & u_y & u_z & -\mathbf{u} \cdot E \\
v_x & v_y & v_z & -\mathbf{v} \cdot E \\
n_x & n_y & n_z & -\mathbf{n} \cdot E \\
0   & 0   & 0   & 1 
\end{bmatrix}.
\]
Przekształcenie do układu kamery wykonujemy jako:
\[
P_{camera} = V \cdot P_{world}.
\]

\subsection{Krok 3: Przelozenie Obiektu z Widoku Kamery na Widok 2D (Rzutowanie Perspektywiczne)}
Aby odwzorować trójwymiarowy widok kamery na dwuwymiarowy obraz, stosujemy macierz rzutowania \( P \) korzystającą z zasad perspektywicznego rzutowania.
Nowe współrzędne \((x', y', z', w')\) są obliczane jako:
\[
P' = P \cdot P_{camera}.
\]
Następnie wykonujemy perspektywiczne dzielenie przez \( w' \):
\[
(x_{ndc}, y_{ndc}, z_{ndc}) = \left(\frac{x'}{w'}, \frac{y'}{w'}, \frac{z'}{w'}\right),
\]
gdzie \( (x_{ndc}, y_{ndc}, z_{ndc}) \) to współrzędne w przestrzeni znormalizowanej (ND). Macierz perspektywiczna \( P \) może mieć postać:
\[
P = \begin{bmatrix}
\frac{f}{a} & 0       & 0                & 0 \\
0       & f       & 0                & 0 \\
0       & 0       & \frac{N+F}{N-F}  & \frac{2NF}{N-F} \\
0       & 0       & -1               & 0 
\end{bmatrix},
\]
gdzie:
\begin{itemize}
    \item \( f = \frac{1}{\tan(\theta/2)} \) – zależny od kąta widzenia \( \theta \),
    \item \( a \) – współczynnik proporcji ekranu,
    \item \( N \) i \( F \) – odległości do płaszczyzny bliskiej (near) i dalszej (far).
\end{itemize}

\subsection{Krok 4: Przelozenie Obiektu na Viewport (Viewport Transformation)}
Ostatecznie, współrzędne z przestrzeni znormalizowanej, gdzie \( x_{ndc}, y_{ndc} \in [-1,1] \), są mapowane do rzeczywistych współrzędnych ekranu.
Dla ekranu o szerokości \( W \) i wysokości \( H \) stosujemy następujące przekształcenie:
\[
x_{screen} = \frac{W}{2} \left( x_{ndc} + 1 \right), \quad y_{screen} = \frac{H}{2} \left( 1 - y_{ndc} \right).
\]
Zapewnia to, że punkt \((-1,-1)\) trafia do lewego dolnego rogu ekranu, a \((1,1)\) do prawego górnego.

\bigskip

Każdy z tych kroków wykorzystuje macierze 4x4 oraz operacje w przestrzeni jednorodnej, co pozwala na efektywne łączenie transformacji oraz zapewnia spójność obliczeń podczas renderowania scen 3D. Metody te stanowią fundament nowoczesnych algorytmów renderowania w grafice komputerowej.

\section{Sterowanie kamera w ukladzie 3D}

Implementacja sterowania kamerą w układzie 3D opiera się na matematycznym modelu transformacji w przestrzeni jednorodnej. W systemie wirtualnej kamery zaimplementowano dwa główne typy ruchu: translację (przesunięcie) oraz rotację (obrót), które są realizowane za pomocą odpowiednich macierzy transformacji.

\subsection{Translacja kamery}

Translacja kamery jest realizowana w lokalnym układzie współrzędnych kamery, co zapewnia intuicyjne sterowanie niezależnie od aktualnej orientacji kamery. Wektor translacji \( \mathbf{t} = [t_x, t_y, t_z] \) jest przekształcany do globalnego układu współrzędnych za pomocą macierzy rotacji kamery.

Dla kamery zorientowanej pod kątem \( \theta_x \) (pitch) i \( \theta_y \) (yaw), wektory kierunkowe w lokalnym układzie kamery są zdefiniowane jako:

\begin{align*}
\mathbf{forward} &= \begin{bmatrix} -\sin(\theta_y) \cos(\theta_x) \\ \sin(\theta_x) \\ -\cos(\theta_y) \cos(\theta_x) \end{bmatrix} \\
\mathbf{right} &= \begin{bmatrix} \cos(\theta_y) \\ 0 \\ -\sin(\theta_y) \end{bmatrix} \\
\mathbf{up} &= \mathbf{right} \times \mathbf{forward}
\end{align*}

Translacja w lokalnym układzie kamery jest realizowana jako kombinacja liniowa tych wektorów:

\[
\mathbf{movement} = \mathbf{right} \cdot t_x + \mathbf{up} \cdot t_y + \mathbf{forward} \cdot t_z
\]

Nowa pozycja kamery \( \mathbf{E}_{new} \) jest obliczana jako:

\[
\mathbf{E}_{new} = \mathbf{E}_{current} + \mathbf{movement}
\]

gdzie \( \mathbf{E}_{current} \) to aktualna pozycja kamery.

\subsection{Rotacja kamery}

Rotacja kamery jest realizowana za pomocą macierzy rotacji wokół lokalnych osi kamery. W systemie zaimplementowano trzy typy rotacji:

\begin{itemize}
    \item \textbf{Pitch} - obrót wokół lokalnej osi X (góra-dół)
    \item \textbf{Yaw} - obrót wokół lokalnej osi Y (lewo-prawo)
    \item \textbf{Roll} - obrót wokół lokalnej osi Z (przechylenie)
\end{itemize}

Macierze rotacji dla poszczególnych osi są zdefiniowane jako:

\begin{align*}
R_x(\theta) &= \begin{bmatrix}
1 & 0 & 0 & 0 \\
0 & \cos(\theta) & -\sin(\theta) & 0 \\
0 & \sin(\theta) & \cos(\theta) & 0 \\
0 & 0 & 0 & 1
\end{bmatrix} \\
R_y(\theta) &= \begin{bmatrix}
\cos(\theta) & 0 & \sin(\theta) & 0 \\
0 & 1 & 0 & 0 \\
-\sin(\theta) & 0 & \cos(\theta) & 0 \\
0 & 0 & 0 & 1
\end{bmatrix} \\
R_z(\theta) &= \begin{bmatrix}
\cos(\theta) & -\sin(\theta) & 0 & 0 \\
\sin(\theta) & \cos(\theta) & 0 & 0 \\
0 & 0 & 1 & 0 \\
0 & 0 & 0 & 1
\end{bmatrix}
\end{align*}

Kombinacja rotacji jest realizowana jako iloczyn macierzy w odpowiedniej kolejności. W systemie zaimplementowano kolejność: yaw \(\rightarrow\) pitch \(\rightarrow\) roll, co zapewnia stabilne i intuicyjne sterowanie:

\[
R_{combined} = R_z \cdot R_x \cdot R_y
\]

Nowa macierz kamery \( C_{new} \) jest obliczana jako:

\[
C_{new} = R_{combined} \cdot C_{current}
\]

gdzie \( C_{current} \) to aktualna macierz kamery.

\subsection{Stabilizacja i ograniczenia}

W systemie zaimplementowano mechanizmy stabilizacji i ograniczeń dla rotacji kamery:

\begin{itemize}
    \item \textbf{Normalizacja yaw} - kąt yaw jest normalizowany do zakresu \([0, 360^\circ)\)
    \item \textbf{Stabilizacja} - funkcja stabilizacji zaokrągla kąty rotacji do najbliższych wartości 90-stopniowych, co ułatwia orientację w przestrzeni
\end{itemize}

\subsubsection{Problem gimbal lock}

Gimbal lock (zawieszenie kardana) to zjawisko występujące w systemach rotacji 3D, w którym traci się jeden stopień swobody. W przypadku kamery, problem ten pojawia się, gdy dwie osie rotacji stają się równoległe, co prowadzi do utraty jednego stopnia swobody. W układzie kardanicznym (eulerowskim), gdy pitch zbliża się do \(\pm 90^\circ\), osie yaw i roll stają się równoległe, co uniemożliwia niezależną kontrolę tych rotacji.

W systemie wirtualnej kamery zaimplementowano dwa podejścia do rozwiązania tego problemu:

\begin{enumerate}
    \item \textbf{Ograniczenie kąta pitch} - kąt pitch jest ograniczony do zakresu \([-179^\circ, 179^\circ]\), co zapobiega osiągnięciu dokładnie \(\pm 180^\circ\), gdzie występuje pełne zawieszenie kardana.
    
    \item \textbf{Kolejność rotacji} - rotacje są aplikowane w określonej kolejności (yaw \(\rightarrow\) pitch \(\rightarrow\) roll), co minimalizuje problem gimbal lock dla większości orientacji kamery.
\end{enumerate}

W obecnej implementacji, ograniczenie kąta pitch do zakresu \([-179^\circ, 179^\circ]\) zapewnia dobry kompromis między swobodą ruchu kamery a stabilnością systemu. Pozwala to na prawie pełny obrót kamery w pionie, jednocześnie unikając problemu gimbal lock.

\section{Podsumowanie Matematyczne}

System wirtualnej kamery opiera się na fundamentalnych zasadach algebry liniowej:
\begin{itemize}
    \item \textbf{Homogeniczne współrzędne:} Pozwalają na łączenie różnych transformacji (translacji, rotacji, skalowania) w jedną operację.
    \item \textbf{Macierze transformacji:} Umożliwiają precyzyjne operacje na punktach poprzez mnożenie macierzy.
    \item \textbf{Twierdzenie Eulera:} Każdy obrót w przestrzeni można sprowadzić do pojedynczego obrotu wokół ustalonej osi, co upraszcza modelowanie ruchu kamery.
    \item \textbf{Rzutowanie perspektywiczne:} Dzielenie przez \(w\) oraz przekształcenie do układu widoku umożliwiają realistyczną projekcję 3D na 2D.
\end{itemize}

\section{Mapowanie sterowania}

System wirtualnej kamery oferuje intuicyjne sterowanie poprzez kombinację klawiatury i myszy. Poniżej przedstawiono szczegółowe mapowanie kontroli:

\subsection{Sterowanie klawiaturą}

\subsubsection{Translacja (Przesunięcie)}
\begin{itemize}
    \item \textbf{W} - przesunięcie w przód (oś Z)
    \item \textbf{S} - przesunięcie w tył (oś Z)
    \item \textbf{A} - przesunięcie w lewo (oś X)
    \item \textbf{D} - przesunięcie w prawo (oś X)
    \item \textbf{Shift} - przesunięcie w górę (oś Y)
    \item \textbf{Ctrl} - przesunięcie w dół (oś Y)
\end{itemize}

\subsubsection{Rotacja (Obrót)}
\begin{itemize}
    \item \textbf{Strzałka w górę} - obrót w górę (pitch)
    \item \textbf{Strzałka w dół} - obrót w dół (pitch)
    \item \textbf{Strzałka w lewo} - obrót w lewo (yaw)
    \item \textbf{Strzałka w prawo} - obrót w prawo (yaw)
    \item \textbf{Alt + Strzałka w lewo} - obrót w lewo (roll)
    \item \textbf{Alt + Strzałka w prawo} - obrót w prawo (roll)
\end{itemize}

\subsubsection{Dodatkowe kontrolki}
\begin{itemize}
    \item \textbf{+} - przybliżenie (zmniejszenie pola widzenia)
    \item \textbf{-} - oddalenie (zwiększenie pola widzenia)
    \item \textbf{R} - reset kamery do pozycji początkowej
    \item \textbf{E} - stabilizacja kamery (wyprostowanie)
    \item \textbf{ESC} - wyjście z aplikacji
\end{itemize}

\subsection{Sterowanie myszą}

\subsubsection{Rotacja (Obrót)}
\begin{itemize}
    \item \textbf{Lewy przycisk myszy + ruch} - obrót kamery w kierunku ruchu myszy
    \begin{itemize}
        \item Ruch w poziomie - obrót wokół osi Y (yaw)
        \item Ruch w pionie - obrót wokół osi X (pitch)
    \end{itemize}
    \item \textbf{Środkowy przycisk myszy + ruch w poziomie} - obrót wokół osi Z (roll)
\end{itemize}

\subsubsection{Zoom}
\begin{itemize}
    \item \textbf{Kółko myszy w górę} - przybliżenie (zmniejszenie pola widzenia)
    \item \textbf{Kółko myszy w dół} - oddalenie (zwiększenie pola widzenia)
\end{itemize}

\subsection{Parametry czułości}

W systemie zaimplementowano następujące parametry czułości:

\begin{itemize}
    \item \textbf{Czułość myszy w poziomie} - określa, jak szybko kamera obraca się w poziomie w odpowiedzi na ruch myszy (w stopniach na piksel)
    \item \textbf{Czułość myszy w pionie} - określa, jak szybko kamera obraca się w pionie w odpowiedzi na ruch myszy (w stopniach na piksel)
    \item \textbf{Czułość roll} - określa, jak szybko kamera obraca się wokół osi Z w odpowiedzi na naciśnięcie klawiszy Alt+strzałki (w stopniach na sekundę)
    \item \textbf{Prędkość ruchu} - określa, jak szybko kamera przesuwa się w przestrzeni w odpowiedzi na naciśnięcie klawiszy WASD (w jednostkach na sekundę)
    \item \textbf{Prędkość zoomu} - określa, jak szybko zmienia się pole widzenia kamery w odpowiedzi na naciśnięcie klawiszy +/- lub kółka myszy
\end{itemize}

\section{Literatura i Źródła}
\begin{enumerate}
    \item Foley, J. D., van Dam, A., Feiner, S. K., \& Hughes, J. F. (1990). \textit{Computer Graphics: Principles and Practice} (2nd ed.). Addison-Wesley.
    \item Takahashi, S. (2006). \textit{The Manga Guide to Linear Algebra}. No Starch Press.
    \item Lengyel, E. (2004). \textit{Mathematics for Game Programming and Computer Graphics}. CRC Press.
    \item Shirley, P., Marschner, S. (2013). \textit{Fundamentals of Computer Graphics} (4th ed.). A K Peters/CRC Press.
    \item \href{https://www.pygame.org/docs/}{Pygame Documentation}.
    \item \href{https://numpy.org/doc/}{NumPy Documentation}.
\end{enumerate}

\end{document}