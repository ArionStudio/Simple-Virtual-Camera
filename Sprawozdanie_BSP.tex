\documentclass[a4paper,12pt]{article}
\usepackage[utf8]{inputenc}
\usepackage[T1]{fontenc}
\usepackage[polish]{babel}
\usepackage{amsmath, amssymb}
\usepackage{hyperref}
\usepackage{geometry}
\usepackage{graphicx}
\geometry{a4paper, margin=1in}
\usepackage{caption}

\title{Sprawozdanie – Algorytm Malarski z BSP}
\author{Adrian Rybaczuk 318483}
\date{\today}

\begin{document}
\maketitle
\tableofcontents
\newpage

\section{Wstęp}

Projekt polega na implementacji systemu wirtualnej kamery z wykorzystaniem algorytmu malarskiego (Painter's Algorithm) wspomaganego przez Binary Space Partitioning (BSP). Celem jest zapewnienie poprawnego renderowania sceny 3D z zachowaniem prawidłowej kolejności rysowania obiektów.

\section{Algorytm Malarski z BSP}

\subsection{Reprezentacja ściany w przestrzeni 3D}

Każda ściana w scenie jest reprezentowana jako zbiór wierzchołków w przestrzeni jednorodnej:
\[
P_i = \begin{bmatrix} x_i \\ y_i \\ z_i \\ 1 \end{bmatrix}, \quad i \in \{1,2,\ldots,n\}
\]

Ściana jest zdefiniowana jako:
\[
F = \{P_1, P_2, \ldots, P_n\}
\]

\subsection{Plaszczyzna dzieląca w BSP}

Dla każdej ściany definiujemy płaszczyznę dzielącą w postaci:
\[
ax + by + cz + d = 0
\]

gdzie wektor normalny płaszczyzny:
\[
\mathbf{n} = \begin{bmatrix} a \\ b \\ c \end{bmatrix}
\]

jest obliczany jako iloczyn wektorowy dwóch krawędzi ściany:
\[
\mathbf{n} = (P_2 - P_1) \times (P_3 - P_1)
\]

\subsection{Podział przestrzeni}

Dla każdej ściany \(F\) i punktu \(P\) w przestrzeni, możemy określić jego położenie względem płaszczyzny dzielącej:

\[
\text{Pozycja}(P) = \begin{cases}
\text{Przed} & \text{gdy } \mathbf{n} \cdot (P - P_1) > 0 \\
\text{Za} & \text{gdy } \mathbf{n} \cdot (P - P_1) < 0 \\
\text{Na} & \text{gdy } \mathbf{n} \cdot (P - P_1) = 0
\end{cases}
\]

\subsection{Struktura drzewa BSP}

Drzewo BSP jest strukturą binarną, gdzie każdy węzeł zawiera:
\begin{itemize}
    \item Ścianę dzielącą
    \item Poddrzewo przed ścianą
    \item Poddrzewo za ścianą
\end{itemize}

Formalnie, drzewo BSP \(T\) jest zdefiniowane jako:
\[
T = \begin{cases}
\emptyset & \text{dla pustego drzewa} \\
(F, T_{przed}, T_{za}) & \text{dla węzła z ścianą } F
\end{cases}
\]

\subsection{Algorytm budowania drzewa BSP}

Dla zbioru ścian \(S = \{F_1, F_2, \ldots, F_n\}\), drzewo BSP jest budowane rekurencyjnie:

\begin{enumerate}
    \item Wybierz ścianę \(F_i\) jako ścianę dzielącą
    \item Podziel pozostałe ściany na zbiory:
    \[
    S_{przed} = \{F_j | F_j \text{ jest przed } F_i\}
    \]
    \[
    S_{za} = \{F_j | F_j \text{ jest za } F_i\}
    \]
    \[
    S_{na} = \{F_j | F_j \text{ jest na } F_i\}
    \]
    \item Rekurencyjnie zbuduj poddrzewa dla \(S_{przed}\) i \(S_{za}\)
\end{enumerate}

\subsection{Algorytm malarski z BSP}

Kolejność rysowania jest określona przez przejście drzewa BSP w porządku in-order, z uwzględnieniem pozycji kamery:

\[
\text{Rysuj}(T, P_{camera}) = \begin{cases}
\emptyset & \text{dla pustego drzewa} \\
\text{Rysuj}(T_{za}, P_{camera}) \cup \{F\} \cup \text{Rysuj}(T_{przed}, P_{camera}) & \text{gdy } P_{camera} \text{ jest przed } F \\
\text{Rysuj}(T_{przed}, P_{camera}) \cup \{F\} \cup \text{Rysuj}(T_{za}, P_{camera}) & \text{gdy } P_{camera} \text{ jest za } F
\end{cases}
\]

\section{Optymalizacje}

\subsection{Przyspieszenie obliczeń}

Dla przyspieszenia obliczeń, wykorzystujemy:
\begin{itemize}
    \item Normalizację wektorów normalnych
    \item Cachowanie wyników testów położenia
    \item Wczesne odrzucanie ścian za kamerą
\end{itemize}

\subsection{Stabilizacja numeryczna}

Aby zapewnić stabilność numeryczną, wprowadzamy epsilon \(\epsilon\):
\[
\text{Pozycja}(P) = \begin{cases}
\text{Przed} & \text{gdy } \mathbf{n} \cdot (P - P_1) > \epsilon \\
\text{Za} & \text{gdy } \mathbf{n} \cdot (P - P_1) < -\epsilon \\
\text{Na} & \text{gdy } |\mathbf{n} \cdot (P - P_1)| \leq \epsilon
\end{cases}
\]

\section{Testy}

\subsection{Test 1: Poprawność kolejności rysowania}
Test sprawdzający poprawność algorytmu BSP w przypadku złożonej sceny z wieloma obiektami. Scena zawiera:
\begin{itemize}
    \item Duży sześcian
    \item Piramidę umieszczoną wewnątrz sześcianu
    \item Cylinder przecinający sześcian
\end{itemize}

Kolejność obiektów na scenie (od najbliższego do najdalszego):
\begin{enumerate}
    \item Cylinder (przecina sześcian, część jest przed nim)
    \item Sześcian (duży, zawiera piramidę)
    \item Piramida (umieszczona wewnątrz sześcianu)
\end{enumerate}

Zalecane ustawienie kamery:
\begin{itemize}
    \item Pozycja: (5, 0, 5) - widok z boku
    \item Kierunek: (-1, 0, 0) - patrząc na scenę z boku
    \item Kąt widzenia: 60 stopni
\end{itemize}

Oczekiwany wynik: Obiekty są rysowane w prawidłowej kolejności, z zachowaniem poprawnej widoczności. Cylinder powinien być poprawnie renderowany względem sześcianu.

\begin{figure}[h]
    \centering
    \begin{minipage}{0.48\textwidth}
        \centering
        \includegraphics[width=\textwidth]{src/Zrzut ekranu 2025-05-18 154308.png}
        \caption{Scena testowa - widok z boku}
    \end{minipage}
    \hfill
    \begin{minipage}{0.48\textwidth}
        \centering
        \includegraphics[width=\textwidth]{src/Zrzut ekranu 2025-05-18 154324.png}
        \caption{Scena przodem}
    \end{minipage}
\end{figure}

\subsection{Test 2: Zachowanie przy zmianie pozycji kamery}
Test sprawdzający zachowanie algorytmu przy dynamicznej zmianie pozycji kamery:
\begin{itemize}
    \item Kamera porusza się wokół złożonej sceny
    \item Sprawdzenie poprawności kolejności rysowania z różnych perspektyw
    \item Weryfikacja płynności przejść między różnymi widokami
\end{itemize}

Kolejność obiektów na scenie:
\begin{enumerate}
    \item Cylinder (na platformie, z prawej strony)
    \item Platforma (pod wszystkimi obiektami)
    \item Piramida (na platformie, z lewej strony)
\end{enumerate}

Zalecane pozycje kamery:
\begin{enumerate}
    \item Widok 1: (0, 0, 10) - widok z przodu
    \item Widok 2: (5, 0, 5) - widok z boku
\end{enumerate}

Oczekiwany wynik: Płynna zmiana kolejności rysowania przy zachowaniu poprawności widoczności.

\begin{figure}[h]
    \centering
    \begin{minipage}{0.48\textwidth}
        \centering
        \includegraphics[width=\textwidth]{src/Zrzut ekranu 2025-05-18 150822.png}
        \caption{Widok 1 - kamera z przodu}
    \end{minipage}
    \hfill
    \begin{minipage}{0.48\textwidth}
        \centering
        \includegraphics[width=\textwidth]{src/Zrzut ekranu 2025-05-18 150913.png}}
        \caption{Widok 2 - kamera z boku}
    \end{minipage}
\end{figure}

\subsection{Test 3: Obsługa przecinających się obiektów}
Test sprawdzający zachowanie algorytmu w przypadku przecinających się obiektów:
\begin{itemize}
    \item Dwa sześciany przecinające się pod kątem 45 stopni
    \item Piramida przecinająca cylinder
    \item Złożony obiekt z wieloma przecinającymi się ścianami
\end{itemize}

Kolejność obiektów na scenie:
\begin{enumerate}
    \item Pierwszy sześcian (przecina się z drugim)
    \item Drugi sześcian (przecina się z pierwszym)
    \item Piramida (przecina cylinder)
    \item Cylinder (przecinany przez piramidę)
\end{enumerate}

Zalecane ustawienie kamery:
\begin{itemize}
    \item Pozycja: (3, 3, 3) - widok ukośny
    \item Kierunek: (-1, -1, -1) - patrząc na przecinające się obiekty
    \item Kąt widzenia: 60 stopni
\end{itemize}

Oczekiwany wynik: Poprawne rozdzielenie przecinających się obiektów i zachowanie prawidłowej kolejności rysowania.

\begin{figure}[h]
    \centering
    \begin{minipage}{0.48\textwidth}
        \centering
        \includegraphics[width=\textwidth]{src/Zrzut ekranu 2025-05-18 154729.png}
        \caption{Scena z przecinającymi się obiektami}
    \end{minipage}
    \hfill
    \begin{minipage}{0.48\textwidth}
        \centering
        \includegraphics[width=\textwidth]{src/Zrzut ekranu 2025-05-18 154751.png}
        \caption{Po przejsciu w bok zmiana kolejnosci}
    \end{minipage}
\end{figure}

\subsection{Test 4: Wydajność przy złożonej scenie}
Test sprawdzający wydajność algorytmu przy dużej liczbie obiektów:
\begin{itemize}
    \item Scena zawierająca więcej obiektów
    \item Różne typy obiektów (sześciany, piramidy, cylindry)
    \item Obiekty umieszczone w różnych odległościach od kamery
\end{itemize}

Układ obiektów na scenie:
\begin{enumerate}
    \item Pierwszy rząd: sześciany (z = 3)
    \item Drugi rząd: piramidy (z = 5)
    \item Trzeci rząd: cylindry (z = 7)
\end{enumerate}

Zalecane ustawienia kamery:
\begin{enumerate}
    \item Widok ogólny: (0, 0, 15) - widok z góry
    \item Widok szczegółowy: (2, 2, 5) - widok z bliska
\end{enumerate}

Oczekiwany wynik: Płynne renderowanie sceny z zachowaniem poprawności kolejności rysowania.

\begin{figure}[h]
    \centering
    \begin{minipage}{0.48\textwidth}
        \centering
        \includegraphics[width=\textwidth]{src/Zrzut ekranu 2025-05-18 153817.png}
        \caption{Widok z przodu ogólny złożonej sceny}
    \end{minipage}
    \hfill
    \begin{minipage}{0.48\textwidth}
        \centering
        \includegraphics[width=\textwidth]{src/Zrzut ekranu 2025-05-18 154152.png}
        \caption{Widok z góry renderwoania}
    \end{minipage}
\end{figure}

\section{Podsumowanie}

Implementacja algorytmu malarskiego z BSP zapewnia:
\begin{itemize}
    \item Poprawną kolejność rysowania obiektów
    \item Efektywne zarządzanie złożonymi scenami
    \item Stabilność numeryczną
    \item Możliwość optymalizacji wydajności
\end{itemize}

\section{Literatura}
\begin{enumerate}
    \item Foley, J. D., van Dam, A., Feiner, S. K., \& Hughes, J. F. (1990). \textit{Computer Graphics: Principles and Practice} (2nd ed.). Addison-Wesley.
    \item BSP Tree FAQ. \url{http://www.faqs.org/faqs/graphics/bsptree-faq/}
    \item Binary Space Partitioning. \url{https://en.wikipedia.org/wiki/Binary_space_partitioning}
    \item Tutorials Point. Binary Space Partitioning Trees in Computer Graphics. \url{https://www.tutorialspoint.com/computer_graphics/computer_graphics_binary_space_partitioning.htm}
    \item Kiciak, P. (2010). Grafika komputerowa I. Uniwersytet Warszawski. \url{https://mst.mimuw.edu.pl/lecture.php?lecture=gk1&part=Ch9}
\end{enumerate}

\end{document} 